\section*{Conclusion}

This paper presented an Open Source automated workflow for writing and preparing \LaTeX~papers for submission in PLOS journals.
The benefits that paper authors and PLOS journals gain from the proposed automated workflow, over using the original PLOS \LaTeX~template directly, were clearly documented.
It then explained how this paper itself has been created using the proposed automated workflow; that it effectively serves as a practical example of the presented approach; while also publishing its source code under an Open Source license.
The software projects that the automated workflow is made of were discussed in detail, and a short demo has been provided.
Following this, a couple of potential issues have been identified: limitations of the presented approach, due to the fairly naive (yet quite effective) \LaTeX~parser that is based on regular expressions; and keeping this automated workflow in sync with any changes to the PLOS publishing practices, its specific \LaTeX~requirements, and updates to the base PLOS \LaTeX~template.
In the end, the author would like to suggest that this automated workflow should be considered for inclusion in the official PLOS \LaTeX~guidelines \cite{PLOS:LaTeX}.

Independent reviews are welcome and enabled by publicly releasing the source code for latex2plos \cite{latex2plos}, template4plos \cite{template4plos}, version4plos \cite{version4plos} and this paper itself \cite{paper:code}; all under an OSI approved BSD Open Source license \cite{OSI:BSD}.
To ease the communication between the author, researchers, users and independent reviewers a public bug tracker has also been provided for each of the projects listed above.
