\section*{Discussion and results}

This paper has been created using the proposed automated workflow, and it effectively serves as a practical example of the presented approach.
For greater transparency the source code for this paper \cite{paper:code} has been publicly released under an OSI approved BSD Open Source license \cite{OSI:BSD}, right from \emph{day one} of its creation.

For paper authors, the proposed automated workflow leads to many practical advantages over using the original PLOS \LaTeX~template directly:

\begin{itemize}
    \item it's much easier to actually start writing your paper as soon as possible; just click/tap the "Use this template" button on the template4plos GitHub project page \cite{template4plos} to get started
    \item it makes it very convenient to use Git/GitHub to track changes while writing your paper, or when responding to reviewers
    \item it automatically checks and prepares your paper for submission in PLOS journals
    \item it minimizes the potential for reference management/inconsistency errors when preparing the paper for submission; for example you may have made changes to your BiBTeX databases or reordered the \verb|\cite| \LaTeX~commands within the paper, but have also managed to forget to update the manually embedded reference information within the paper - causing out-of-order references, or even displaying stale data
    \item it allows you to use figures in the standard \LaTeX~manner while writing your paper; it will then automatically copy these figures, transform them into the TIFF image format and finally remove/comment-out them from the exported \code{.tex} manuscript (and its PDF), as per PLOS requirements
    \item it allows you to store code listings in external files while writing your paper; it will then automatically embed these files into the exported \code{.tex} manuscript
    \item it allows you to use the standard multiple \LaTeX~files workflow while writing your paper; it will then automatically combine them into a single, cohesive \code{.tex} file, as PLOS does not allow submission of multiple \code{.tex} files
    \item preparing your paper for submission is no longer a "destructive" process; the act of automated preparation will never disturb any of your original manuscript files and can also be run as often as you'd like
    \item you don't even have to bother with installing and configuring a \LaTeX~distribution on your system; everything can be done through the Dockerized version of the proposed automated workflow instead
\end{itemize}
PLOS journals will see a practical benefit of this approach as well, in the form of a higher level of consistency in their submitted/published papers and a lower frequency of \LaTeX-related support requests.

It's important to recognize and discuss the limitations of the presented approach, namely that the latex2plos project uses a fairly naive (yet quite effective) \LaTeX~parser that is based on regular expressions.
Because of this, latex2plos will not be able to process \LaTeX~code with extensive "clever" macros as well as the native \LaTeX~toolkit.
The author would prefer to use a true \LaTeX~parser instead of a regular expressions based solution, but at the moment of this writing he's not aware of a Python library providing such services (basically a \LaTeX~to \LaTeX~source code trans-compiler) in a feature complete manner.

In the end, it's vital to examine the need of keeping this automated workflow in sync with any changes to the PLOS publishing practices, its specific \LaTeX~requirements, and updates to the base PLOS \LaTeX~template.
After all, every realistically used software or process evolves over time, and instead of trying to fight such changes the author of this paper wishes to embrace them.
To this effect, the author has also created a small Python powered library and console tool, named version4plos \cite{version4plos}, for automated tracking of new PLOS \LaTeX~template versions, as they're released on the official PLOS \LaTeX~guidelines page \cite{PLOS:LaTeX}.
He has also setup a cron job on one of his servers to run the version4plos tool daily and report back (via email) if an update has been made to the PLOS \LaTeX~template.
Once a new PLOS \LaTeX~template release has been detected the projects presented in this paper will also be updated, as needed, to keep them in sync with the newly released template.

The author of this paper has over 14 years of experience in producing and supporting Open Source Software.
During this period he became a Python Software Foundation contributing member with 30+ Open Source projects created, 27 Python packages published on the Python Package Index and over 500 registered contributions on GitHub in 2019 alone \cite{Maric:GitHub, Maric:PyPI}.
He has also created and is managing the official country level mirrors for CTAN (Comprehensive \TeX~Archive Network), Ubuntu, Arch Linux, MariaDB, CentOS, Debian and MX Linux.
Because of this, it's clearly visible that he is more than capable of supporting the Open Source projects presented in this paper in the long term; by adding new features, fixing bugs and keeping them up to date with the current PLOS \LaTeX~template.
