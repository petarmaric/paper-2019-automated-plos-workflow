\section*{Project \code{template4plos}}

The template4plos project \cite{template4plos} provides a reference Open Source Software implementation for the proposed automated workflow.
To achieve this, it implements a modified version of the PLOS \LaTeX~template, preconfigured for automated preparation of your \LaTeX~paper for submission in PLOS journals.
For greater transparency the source code of the template4plos project \cite{template4plos} has been publicly released under an OSI approved BSD Open Source license \cite{OSI:BSD}, right from \emph{day one} of its creation.

This project is published in the form of a GitHub template repository \cite{GitHub:help:repository_template}, which allows its users to simply and quickly create GitHub repositories for their own papers, while keeping the original directory and file layout of the template repository.
It also includes an extensive installation and usage manual, that's distributed with each created paper repository as well.

The template4plos project is composed of three distinct layers, going from inward to outward: a modified version of the PLOS \LaTeX~template, Makefile powered automated workflow, and the utility Dockerized layer.

\subsection*{Layer 0: original PLOS \LaTeX~template}

This layer has been included purely for completeness sake, and is thus not formally counted in the template4plos layer list.
While technically a (innermost) layer, it's still an externally provided resource \cite{PLOS:LaTeX} from outside the system -- template4plos doesn't introduce anything new to it, at least not directly.

\subsection*{Layer 1: modified version of the PLOS \LaTeX~template}

The modified version of the original PLOS \LaTeX~template is the only layer that's purely static in nature, as it focuses solely on the structure of the \LaTeX~document and doesn't model any of the proposed automated workflow's dynamic behavior.
It's setup in such a way to allow for minimum bootstrap time when beginning work on your paper.

This layer mimics the original PLOS \LaTeX~template as close as possible, with a few notable exceptions:
\begin{itemize}
    \item it's cut up into multiple files, to allow for the standard multiple \LaTeX~files workflow -- the paper itself (\code{paper.tex}), its preamble (\code{preamble.tex}), authors (\code{head/authors.tex}), title (\code{head/title.tex}), abstract (\code{head/abstract.tex}) and the sections (\code{sections/*.tex})
    \item it uses a BiBTeX database (\code{references.bib}) for reference management, instead of manually embedding reference information into the \LaTeX~document directly
    \item it introduces additional utility \LaTeX~packages into the template (such as \code{cleveref}, \code{float}, \code{verbatim}, \code{listings}, \code{inconsolata}) as well as a few utility \LaTeX~commands (such as \code{code}), for improved writing experience
    \item it adds a new section "\code{latex2plos} tests", to demonstrate the commonly used \LaTeX~features that were not present in the original template, and verify that \code{latex2plos} is able to handle them properly
\end{itemize}

This should also be the only layer that requires updating when a new PLOS \LaTeX~template version is released.

\subsection*{Layer 2: Makefile powered automated workflow}

The Makefile layer provides the dynamic part of the proposed automated workflow, and is composed of the following Makefile tasks/targets:
\begin{itemize}
    \item \code{help}, displays template4plos help message
    \item \code{build}, uses \code{pdftex} and \code{latexmk} to build the paper
    \item \code{export}, uses \code{latex2plos} to prepare the paper for journal submission
    \item \code{clean-built}, removes the papers build directory
    \item \code{clean-exported}, removes the papers export directory
    \item \code{clean}, removes both the build and export directories
    \item \code{view-built}, views the papers built PDF file, running \code{build} beforehand if needed
    \item \code{view-exported}, views the papers exported PDF file, running \code{export} beforehand if needed
    \item \code{view}, views both the built and exported PDF files
\end{itemize}

Once you've created and cloned your paper's GitHub repository run the following command to setup the Makefile layer:
\begin{verbatim}
    $ pipenv install
\end{verbatim}

From there on simply write your paper, while keeping in mind you can edit any of its files freely; such as adding/removing sections, adding/removing entries from the \code{references.bib} BiBTeX database, adding more \LaTeX~packages to \code{preamble.tex}, or changing other options.
But please remember to build (and check) your paper often as you write:
\begin{verbatim}
    $ pipenv run make view-built
\end{verbatim}

\noindent
Finally, run the following command to prepare your paper for journal submission:
\begin{verbatim}
    $ pipenv run make view-exported
\end{verbatim}
and then submit files from the export directory to the journal.

\subsection*{Layer 3: Docker powered automated workflow}

The utility Dockerized layer wraps around the Makefile powered workflow, for people who do not wish to bother with installing and configuring a \LaTeX~distribution on their system.
It cleanly replicates the features of the Makefile powered workflow, while also providing an additional means of software-based reproducibility when building a paper -- the template4plos Docker images are effectively set in stone, and once (automatically) built and published on Docker Hub they are never rebuilt again \cite{template4plos:DockerHub:tags}.

To use the dockerized version of template4plos, simply replace your existing \code{make} calls:

\begin{verbatim}
    $ pipenv run make [TARGET] ...
\end{verbatim}
with:
\begin{verbatim}
    $ ./dockerized.sh [TARGET] ...
\end{verbatim}

\noindent
For instance, run the following command to show template4plos help:
\begin{verbatim}
    $ ./dockerized.sh help
\end{verbatim}

\noindent
Or, run the following command to build your paper:
\begin{verbatim}
    $ ./dockerized.sh build
\end{verbatim}

\noindent
Finally, run the following command to prepare your paper for journal submission:
\begin{verbatim}
    $ ./dockerized.sh export
\end{verbatim}
and then submit files from the export directory to the journal.
