\section*{Project \code{latex2plos}}

The latex2plos project \cite{latex2plos} is a Python \cite{Python} powered library/console tool for automated preparation of your \LaTeX~paper for submission in PLOS journals, and is at the core of the proposed automated workflow.
It's designed to address the issues discussed in the "Introduction" section, and resolve them effectively.
It does so by taking the original \LaTeX~manuscript as its input, transforming the said manuscript and its resources (such as figures, reference information, and code listings) into a submission-ready version of the paper, that conforms to the official PLOS figure and \LaTeX~guidelines \cite{PLOS:Figures, PLOS:LaTeX}.
For greater transparency the source code of the latex2plos project \cite{latex2plos} has been publicly released under an OSI approved BSD Open Source license \cite{OSI:BSD}, right from \emph{day one} of its creation.

The latex2plos project is composed of three distinct components: a \LaTeX~parser, transformers and console shell interface.

The \LaTeX~parser is based on regular expressions.
For each line of input the parser first checks if it starts with a commented-out \LaTeX~command, and if not it then looks for characteristic markers that match against a specific transformer.
If no such marker is found the line is copied as-is, otherwise an appropriate transformer is invoked where it's able to manipulate the \LaTeX~document directly and/or its referenced resources.
The document itself is processed and streamed through on the fly, for maximum efficiency.

Transformers manipulate the \LaTeX~document and its referenced resources (such as figures, reference information, and code listings), to achieve compliance with the official PLOS figure and \LaTeX~guidelines \cite{PLOS:Figures, PLOS:LaTeX}.
All transformers are implemented as plugins.
The dynamic plugin subsystem, based on the \code{simple\_plugins} project \cite{Project:simple_plugins:CodeRepository}, provides automatic transformer plugin discovery and registration.

The author of this paper is a firm believer in the API-centric software development doctrine and, like most of his other Open Source projects, latex2plos is first and foremost a Python library -- with its public API exposed through a console shell interface as well.

This remainder of this section has been carried over from the template4plos project \cite{template4plos}, to demonstrate the commonly used \LaTeX~features that are not present in the original PLOS \LaTeX~template, and verify that the latex2plos project is able to handle them properly.
While doing that, it also documents the individual transformer plugins.

\subsection*{\code{verbatiminput}}

\code{latex2plos.transformers.VerbatimInputTransformer} will transform a \verb|\verbatiminput| file inclusion into its equivalent \verb|\begin{verbatim}...\end{verbatim}| call, as in the example below \cite{Project:friendly_name_mixin:CodeRepository}:

\verbatiminput{listings/friendly_name_mixin.py}

\subsection*{\code{bibliography}}

\code{latex2plos.transformers.BibliographyTransformer} will transform a \verb|\bibliography| file inclusion into its equivalent \verb|\begin{thebibliography}...\end{thebibliography}| call, as demonstrated by this paper (\code{paper.tex}) and its BiBTeX database (\code{references.bib}).

This is done as PLOS does not allow submission of BiBTeX databases, and instead requires the reference information to be embedded in the paper directly \cite{PLOS:LaTeX}.

\subsection*{\code{lstinputlisting}}

\code{latex2plos.transformers.InputListingTransformer} will copy over the files referenced by \verb|\lstinputlisting| into the papers export directory, as in the examples below:

\lstinputlisting[language=Python,caption={\code{friendly\_name\_mixin} project source code \cite{Project:friendly_name_mixin:CodeRepository}}]{listings/friendly_name_mixin.py}

\lstinputlisting[language=Python,caption={\code{friendly\_name\_mixin} project unit tests \cite{Project:friendly_name_mixin:CodeRepository}}]{listings/tests.py}

\pagebreak
\subsection*{\code{includegraphics}}

\code{latex2plos.transformers.IncludeGraphicsTransformer} will copy over the figures referenced by \verb|\includegraphics| into the papers export directory, as shown in \cref{fig:petar-avatar,fig:3-clamped-free-beam-decimal-precision-errors,fig:barbero-viscoelastic-natural-frequencies,fig:barbero-viscoelastic-damage-analysis}.
It will also transform all figures into the TIFF image format (LZW compression, at 300 dpi) and remove/comment-out them from the papers exported PDF, as per PLOS requirements \cite{PLOS:LaTeX, PLOS:Figures}.

\begin{figure}[H]
    \centering
    \includegraphics[height=6cm]{figures/petar-gravatar.png}
    \caption{Petar Marić gravatar (PNG image format) \cite{Maric:Gravatar}}
    \label{fig:petar-avatar}
\end{figure}

\begin{figure}[H]
    \centering
    \includegraphics[width=\linewidth]{figures/decimal-precision-error@3-clamped-free-beam.eps}
    \caption{Maximum root-finding errors (across all 100 modes) of clamped-free beam's (LJ=3) characteristic function (EPS image format) \cite{Maric:2017}}
    \label{fig:3-clamped-free-beam-decimal-precision-errors}
\end{figure}

\begin{figure}[H]
    \centering
    \includegraphics[width=\linewidth]{figures/barbero-viscoelastic@local-p1.jpg}
    \caption{Viscoelastic natural frequencies (JPEG image format) \cite{Milasinovic:2018}}
    \label{fig:barbero-viscoelastic-natural-frequencies}
\end{figure}

\begin{figure}[H]
    \centering
    \includegraphics[width=\linewidth]{figures/barbero-viscoelastic@fsm-damage-analysis.pdf}
    \caption{\code{fsm\_damage\_analysis} project example experiment report (PDF image format) \cite{Project:fsm_damage_analysis:CodeRepository}}
    \label{fig:barbero-viscoelastic-damage-analysis}
\end{figure}

\subsection*{Multiple \LaTeX~files workflow}

\code{latex2plos.transformers.IncludeTransformer} will embed the contents of files referenced by \verb|\include| into the exported paper directly, whereas \code{latex2plos.transformers.InputTransformer} will do the same for \verb|\input|, as demonstrated by this paper (\code{paper.tex}) and its referenced \code{.tex} files (\code{preamble.tex},  \code{head/*.tex} and \code{sections/*.tex}).

This is done as PLOS does not allow submission of multiple \code{.tex} files, and instead requires them to be combined into a single, cohesive \code{.tex} file \cite{PLOS:LaTeX}.
